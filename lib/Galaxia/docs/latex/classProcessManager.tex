\section{Process\-Manager Class Reference}
\label{classProcessManager}\index{ProcessManager@{ProcessManager}}
Inheritance diagram for Process\-Manager:\begin{figure}[H]
\begin{center}
\leavevmode
\includegraphics[width=63pt]{classProcessManager__inherit__graph}
\end{center}
\end{figure}
Collaboration diagram for Process\-Manager:\begin{figure}[H]
\begin{center}
\leavevmode
\includegraphics[width=63pt]{classProcessManager__coll__graph}
\end{center}
\end{figure}
\subsection*{Public Member Functions}
\begin{CompactItemize}
\item 
{\bf Process\-Manager} (\$db)
\item 
{\bf activate\_\-process} (\$p\-Id)
\item 
{\bf deactivate\_\-process} (\$p\-Id)
\item 
{\bf serialize\_\-process} (\$p\-Id)
\item 
{\bf unserialize\_\-process} (\$xml)
\item 
{\bf import\_\-process} (\$data)
\item 
{\bf new\_\-process\_\-version} (\$p\-Id, \$minor=true)
\item 
{\bf process\_\-name\_\-exists} (\$name, \$version)
\item 
{\bf get\_\-process} (\$p\-Id)
\item 
{\bf list\_\-processes} (\$offset, \$max\-Records, \$sort\_\-mode, \$find, \$where='')
\item 
{\bf invalidate\_\-process} (\$pid)
\item 
{\bf remove\_\-process} (\$p\-Id)
\item 
{\bf replace\_\-process} (\$p\-Id, \$vars, \$create=true)
\item 
{\bf \_\-rec\_\-copy} (\$dir1, \$dir2)\label{classProcessManager_a13}

\item 
{\bf \_\-start\_\-element\_\-handler} (\$parser, \$element, \$attribs)\label{classProcessManager_a14}

\item 
{\bf \_\-end\_\-element\_\-handler} (\$parser, \$element)\label{classProcessManager_a15}

\item 
{\bf \_\-data\_\-handler} (\$parser, \$data)\label{classProcessManager_a16}

\end{CompactItemize}
\subsection*{Public Attributes}
\begin{CompactItemize}
\item 
{\bf \$parser}\label{classProcessManager_o0}

\item 
{\bf \$tree}\label{classProcessManager_o1}

\item 
{\bf \$current}\label{classProcessManager_o2}

\item 
{\bf \$buffer}\label{classProcessManager_o3}

\end{CompactItemize}


\subsection{Detailed Description}
This class is used to add,remove,modify and list processes. 



Definition at line 9 of file Process\-Manager.php.

\subsection{Constructor \& Destructor Documentation}
\index{ProcessManager@{Process\-Manager}!ProcessManager@{ProcessManager}}
\index{ProcessManager@{ProcessManager}!ProcessManager@{Process\-Manager}}
\subsubsection{\setlength{\rightskip}{0pt plus 5cm}Process\-Manager::Process\-Manager (\$ {\em db})}\label{classProcessManager_a0}


Constructor takes a PEAR::Db object to be used to manipulate roles in the database. 

Definition at line 19 of file Process\-Manager.php.

\subsection{Member Function Documentation}
\index{ProcessManager@{Process\-Manager}!activate_process@{activate\_\-process}}
\index{activate_process@{activate\_\-process}!ProcessManager@{Process\-Manager}}
\subsubsection{\setlength{\rightskip}{0pt plus 5cm}Process\-Manager::activate\_\-process (\$ {\em p\-Id})}\label{classProcessManager_a1}


Sets a process as active 

Definition at line 31 of file Process\-Manager.php.\index{ProcessManager@{Process\-Manager}!deactivate_process@{deactivate\_\-process}}
\index{deactivate_process@{deactivate\_\-process}!ProcessManager@{Process\-Manager}}
\subsubsection{\setlength{\rightskip}{0pt plus 5cm}Process\-Manager::deactivate\_\-process (\$ {\em p\-Id})}\label{classProcessManager_a2}


De-activates a process 

Definition at line 42 of file Process\-Manager.php.\index{ProcessManager@{Process\-Manager}!get_process@{get\_\-process}}
\index{get_process@{get\_\-process}!ProcessManager@{Process\-Manager}}
\subsubsection{\setlength{\rightskip}{0pt plus 5cm}Process\-Manager::get\_\-process (\$ {\em p\-Id})}\label{classProcessManager_a8}


Gets a process by p\-Id. Fields are returned as an asociative array 

Definition at line 370 of file Process\-Manager.php.\index{ProcessManager@{Process\-Manager}!import_process@{import\_\-process}}
\index{import_process@{import\_\-process}!ProcessManager@{Process\-Manager}}
\subsubsection{\setlength{\rightskip}{0pt plus 5cm}Process\-Manager::import\_\-process (\$ {\em data})}\label{classProcessManager_a5}


Creates a process from the process data structure, if you want to convert an XML to a process then use first unserialize\_\-process and then this method. 

Definition at line 231 of file Process\-Manager.php.\index{ProcessManager@{Process\-Manager}!invalidate_process@{invalidate\_\-process}}
\index{invalidate_process@{invalidate\_\-process}!ProcessManager@{Process\-Manager}}
\subsubsection{\setlength{\rightskip}{0pt plus 5cm}Process\-Manager::invalidate\_\-process (\$ {\em pid})}\label{classProcessManager_a10}


Marks a process as an invalid process 

Definition at line 415 of file Process\-Manager.php.\index{ProcessManager@{Process\-Manager}!list_processes@{list\_\-processes}}
\index{list_processes@{list\_\-processes}!ProcessManager@{Process\-Manager}}
\subsubsection{\setlength{\rightskip}{0pt plus 5cm}Process\-Manager::list\_\-processes (\$ {\em offset}, \$ {\em max\-Records}, \$ {\em sort\_\-mode}, \$ {\em find}, \$ {\em where} = '')}\label{classProcessManager_a9}


Lists processes (all processes) 

Definition at line 382 of file Process\-Manager.php.\index{ProcessManager@{Process\-Manager}!new_process_version@{new\_\-process\_\-version}}
\index{new_process_version@{new\_\-process\_\-version}!ProcessManager@{Process\-Manager}}
\subsubsection{\setlength{\rightskip}{0pt plus 5cm}Process\-Manager::new\_\-process\_\-version (\$ {\em p\-Id}, \$ {\em minor} = true)}\label{classProcessManager_a6}


Creates a new process based on an existing process changing the process version. By default the process is created as an unactive process and the version is by default a minor version of the process.

\begin{Desc}
\item[{\bf Todo}]copy process activities and so \end{Desc}


Definition at line 318 of file Process\-Manager.php.\index{ProcessManager@{Process\-Manager}!process_name_exists@{process\_\-name\_\-exists}}
\index{process_name_exists@{process\_\-name\_\-exists}!ProcessManager@{Process\-Manager}}
\subsubsection{\setlength{\rightskip}{0pt plus 5cm}Process\-Manager::process\_\-name\_\-exists (\$ {\em name}, \$ {\em version})}\label{classProcessManager_a7}


This function can be used to check if a process name exists, note that this is NOT used by replace\_\-process since that function can be used to create new versions of an existing process. The application must use this method to ensure that processes have unique names. 

Definition at line 360 of file Process\-Manager.php.\index{ProcessManager@{Process\-Manager}!remove_process@{remove\_\-process}}
\index{remove_process@{remove\_\-process}!ProcessManager@{Process\-Manager}}
\subsubsection{\setlength{\rightskip}{0pt plus 5cm}Process\-Manager::remove\_\-process (\$ {\em p\-Id})}\label{classProcessManager_a11}


Removes a process by p\-Id 

Definition at line 424 of file Process\-Manager.php.\index{ProcessManager@{Process\-Manager}!replace_process@{replace\_\-process}}
\index{replace_process@{replace\_\-process}!ProcessManager@{Process\-Manager}}
\subsubsection{\setlength{\rightskip}{0pt plus 5cm}Process\-Manager::replace\_\-process (\$ {\em p\-Id}, \$ {\em vars}, \$ {\em create} = true)}\label{classProcessManager_a12}


Updates or inserts a new process in the database, \$vars is an asociative array containing the fields to update or to insert as needed. \$p\-Id is the process\-Id 

Definition at line 463 of file Process\-Manager.php.\index{ProcessManager@{Process\-Manager}!serialize_process@{serialize\_\-process}}
\index{serialize_process@{serialize\_\-process}!ProcessManager@{Process\-Manager}}
\subsubsection{\setlength{\rightskip}{0pt plus 5cm}Process\-Manager::serialize\_\-process (\$ {\em p\-Id})}\label{classProcessManager_a3}


Creates an XML representation of a process. 

Definition at line 53 of file Process\-Manager.php.\index{ProcessManager@{Process\-Manager}!unserialize_process@{unserialize\_\-process}}
\index{unserialize_process@{unserialize\_\-process}!ProcessManager@{Process\-Manager}}
\subsubsection{\setlength{\rightskip}{0pt plus 5cm}Process\-Manager::unserialize\_\-process (\$ {\em xml})}\label{classProcessManager_a4}


Creates a process PHP data structure from its XML representation 

Definition at line 135 of file Process\-Manager.php.

The documentation for this class was generated from the following file:\begin{CompactItemize}
\item 
Process\-Manager.php\end{CompactItemize}
